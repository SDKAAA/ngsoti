\section*{Executive Summary}
\addcontentsline{toc}{section}{Executive Summary}

This deliverable adheres to the definition outlined in the grant agreement, as stated below:

\textbf{ Interim report with key activities of the NGSOTI infrastructure, including its deployment and usage status}.


\chapter{Introduction}
Replace with actual content.

\chapter{NGSOTI trainings}
Some EU regulations, such as NIS2 and DORA, have created a significant boost in the demand for NGSOTI training sessions. These sessions are designed to ensure organizations are well-prepared to meet the requirements of these regulations, which are often enforced starting in 2025. A key focus of these regulations is the establishment of local SOC (Security Operations Center).

The NIS2 Directive mandates that regulated entities implement robust incident response capabilities and maintain effective security operations. Specifically, organizations must establish comprehensive incident handling policies that define roles, responsibilities, and procedures for detecting, analyzing, and responding to security incidents. These policies also cover post-incident activities such as recovery, documentation, and reporting.

Furthermore, the directive emphasizes the importance of continuous monitoring and logging of network and information systems to promptly detect and address potential threats. Entities are expected to:

\begin{itemize}
    \item Implement automated monitoring where feasible.
    \item Regularly review logs to identify unusual activities.
    \item Ensure accurate time synchronization across systems to facilitate effective incident analysis.
\end{itemize}

By adhering to these requirements, organizations can enhance their resilience against cyber threats and ensure compliance with the NIS2 Directive. Article 21, under point (b), explicitly requires the establishment of incident handling capabilities.

Importantly, NIS2 extends its scope to include entities such as SMEs (Small and Medium Enterprises) involved in the supply chains of Operators of Essential Services (OES) or Digital Service Providers (DSP), which were not regulated under the original NIS Directive. These entities are now required to set up local incident response capabilities.

Given this regulatory landscape, the NGSOTI framework provided an excellent opportunity to conduct targeted incident response training. These efforts were aimed at equipping participants with the skills and knowledge required to comply with these critical regulations. The incident response trainings conducted are detailed in the table below:


\begin{table}
    \begin{tabular}{lllll}
        \hline
        Date & Training  & Number of participants & Sector & Target audiance\\
        \hline
        2024-07-05 & Kunai & 12 & Multiple & Open source enthusiasts \\
        2024-11-08 & Incident Response & 9 & Finance & Security professionals\\
        2024-11-10 & Incident Response & 4 & Finance & Security professionals\\
        2024-11-11 & Incident Response & 4 & Finance & Security professionals\\
        2024-11-23 & Forensic & 4 & Finance & Security professionals\\
        2024-11-24 & Forensic & 4 & Finance & Security professionals\\
        2024-12-11 & Forensic & xx & Education & Engineering students \\
        2024-12-16 & MISP threat sharing & xx & Education & Engineering students \\
        \hline
    \end{tabular}
\end{table}


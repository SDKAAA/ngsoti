\section*{Executive Summary}
\addcontentsline{toc}{section}{Executive Summary}

This deliverable adheres to the definition outlined in the grant agreement, as stated below:

\textbf{ Interim report with key activities of the NGSOTI infrastructure, including its deployment and usage status}.


\chapter{Introduction}
Replace with actual content.

\chapter{NGSOTI setup and operation}
In the NGSOTI project, the consortium closely monitored the behavior and requirements of its user communities, ensuring the development and customization of tools tailored to support efficient SOC operations. These tools are detailed in the following section.

\section{Optimized Tools for Streamlined SOC Operations}


\subsection{Tenzir}


\subsection{SkillAegis}
SkillAegis is an excellent outcome of the NGSOTI project, aligning seamlessly with its mission to train the next generation of Security Operations Center (SOC) operators. While the project emphasizes creating operational infrastructures and fostering hands-on training environments, SkillAegis addresses the critical human element by providing a platform for scenario-based training.
SkillAegis equips trainees with practical skills to navigate these challenges by simulating real-world incidents in a controlled, dynamic environment.
This tool complements the project's vision by enhancing traditional training methods, including academic curricula and industry-led guest lectures, with interactive, real-data-driven cyber range exercises. Hosted by CIRCL and supported by Restena's network interconnectivity, the infrastructure enables SkillAegis to play a pivotal role in preparing SOC operators for future challenges while fostering a collaborative and robust educational ecosystem.

On August 14, 2024, the release of \textbf{SkillAegis v1.0.0} was announced on misp-project.org. SkillAegis is an open-source platform developed as part of the NGSOTI project to advance cybersecurity training and capacity building. SkillAegis is designed to enhance cyber threat intelligence training through realistic, scenario-based exercises, enabling participants to simulate and respond to real-world cyber incidents.

\subsubsection{Key Features of SkillAegis}
\begin{itemize}
    \item \textbf{Scenario Creation:} Trainers can design customized exercises with specific learning objectives, simulating various cyber incidents to develop practical skills in threat intelligence and information management.
    \item \textbf{Exercise Execution \& Monitoring:} The platform allows trainers to deploy and oversee scenarios in real-time, using a live dashboard to track participant progress and provide immediate feedback.
\end{itemize}

\subsubsection{Components of the SkillAegis Platform}
\begin{itemize}
    \item \textbf{SkillAegis Main Application:} Serves as the core component, managing scenario execution and integrating ready-to-use training modules.
    \item \textbf{SkillAegis Editor:} Enables trainers to create new exercises, including metadata, injects (tasks), execution order, and evaluation criteria. It also features an Inject Tester to optimize scenarios.
    \item \textbf{SkillAegis Dashboard:} Facilitates training session execution and provides real-time insights into participant progress, including completion status and logs of user actions.
\end{itemize}

\subsubsection{Integration with MISP}
SkillAegis was specifically developed to integrate with MISP (Threat Sharing Platform), enhancing training quality. Injects can directly interact with data within MISP, querying and monitoring user activity. To enable this, SkillAegis connects to a training MISP instance via a valid site-admin API (Application Programming Interface) Key.

\subsubsection{Availability and Licensing}
SkillAegis is freely available as open-source software under the AGPLv3 license, reflecting its EU project origins and commitment to supporting the cybersecurity community. The platform can be downloaded and utilized at no cost.

For more information and access to SkillAegis, visit the \href{https://www.misp-project.org/2024/08/14/SkillAegis-v1.0.0.html/}{official announcement}.


\subsection{MISP training Infrastructure}



\chapter{NGSOTI trainings}
Some EU regulations, such as NIS2 and DORA, have created a significant boost in the demand for NGSOTI training sessions. These sessions are designed to ensure organizations are well-prepared to meet the requirements of these regulations, which are often enforced starting in 2025. A key focus of these regulations is the establishment of local SOC (Security Operations Center).

The NIS2 Directive mandates that regulated entities implement robust incident response capabilities and maintain effective security operations. Specifically, organizations must establish comprehensive incident handling policies that define roles, responsibilities, and procedures for detecting, analyzing, and responding to security incidents. These policies also cover post-incident activities such as recovery, documentation, and reporting.

Furthermore, the directive emphasizes the importance of continuous monitoring and logging of network and information systems to promptly detect and address potential threats. Entities are expected to:

\begin{itemize}
    \item Implement automated monitoring where feasible.
    \item Regularly review logs to identify unusual activities.
    \item Ensure accurate time synchronization across systems to facilitate effective incident analysis.
\end{itemize}

By adhering to these requirements, organizations can enhance their resilience against cyber threats and ensure compliance with the NIS2 Directive. Article 21, under point (b), explicitly requires the establishment of incident handling capabilities.

Importantly, NIS2 extends its scope to include entities such as SMEs (Small and Medium Enterprises) involved in the supply chains of Operators of Essential Services (OES) or Digital Service Providers (DSP), which were not regulated under the original NIS Directive. These entities are now required to set up local incident response capabilities.

Given this regulatory landscape, the NGSOTI framework provided an excellent opportunity to conduct targeted incident response training. These efforts were aimed at equipping participants with the skills and knowledge required to comply with these critical regulations. The incident response trainings conducted are detailed in the table below:


\begin{table}
    \begin{tabular}{lllll}
        \hline
        Date & Training  & Number of participants & Sector & Target audiance\\
        \hline
        2024-07-05 & Kunai & 12 & Multiple & Open source enthusiasts \\
        2024-11-08 & Incident Response & 9 & Finance & Security professionals\\
        2024-11-10 & Incident Response & 4 & Finance & Security professionals\\
        2024-11-11 & Incident Response & 4 & Finance & Security professionals\\
        2024-11-23 & Forensic & 4 & Finance & Security professionals\\
        2024-11-24 & Forensic & 4 & Finance & Security professionals\\
        2024-12-11 & Forensic & xx & Education & Engineering students \\
        2024-12-16 & MISP threat sharing & xx & Education & Engineering students \\
        \hline
    \end{tabular}
\end{table}

